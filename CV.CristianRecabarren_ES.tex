\documentclass[a4paper,10pt]{article}
\usepackage[utf8]{inputenc}
\usepackage[spanish]{babel}
\usepackage{enumitem}
\usepackage{geometry}
\usepackage{fancyhdr}
\usepackage{titlesec}
\usepackage{xcolor}

\geometry{top=1in, bottom=1in, left=1in, right=1in}

\pagestyle{fancy}
\fancyhf{}
\renewcommand{\headrulewidth}{0pt}

\titleformat{\section}
  {\normalfont\large\bfseries\color{blue}}
  {}
  {0em}
  {}[\titlerule]

\begin{document}

\begin{center}
    \textbf{\LARGE Cristian Rubén Recabarren Madrid}
\end{center}

\section*{Información de Contacto}
\begin{tabular}{p{3cm} p{10cm}}
    Teléfono: & +56 9 9834 7770 \\
    Correo Electrónico: & crecabar\_cl@me.com \\
\end{tabular}

\section*{Perfil}
Ingeniero en Civil en Computación con una sólida experiencia en desarrollo de software y un historial comprobado en liderar y gestionar equipos para la entrega exitosa de proyectos. Comprometido con la mejora continua y la búsqueda de soluciones óptimas.

\section*{Experiencia Laboral}
\subsection*{Ingeniero de Software, Comisión Nacional de Energía de Chile}
\begin{itemize}[left=0em]
    \item Exitosamente lideré equipos multifuncionales en la mejora de los scripts ETL de "Energía Región" en Python.
    \item Jugé un papel crucial en la mejora de la aplicación "Informe Diario del Sector Energético" como desarrollador.
    \item Actué como líder de proyecto en la Plataforma de Monitoreo del Mercado Eléctrico.
    \item Desarrollé e implementé varios procesos para la nueva plataforma de procedimientos en línea.
\end{itemize}

\subsection*{Ingeniero de Software, Ministerio de Energía de Chile}
\begin{itemize}[left=0em]
    \item Diseñé y gestioné con éxito el desarrollo de la nueva versión de la plataforma de Gestión de Energía utilizando C# y ASP.NET como parte de la iniciativa de Gobierno Eficiente.
    \item Lideré y desarrollé cambios importantes en la plataforma de trámites en línea para hacerla amigable para la Oficina de Información, Reclamos y Sugerencias del Ciudadano. Estas mejoras se implementaron desde la versión 1 hasta la versión actual 2, que se utiliza en todas las plataformas de trámites en línea del gobierno.
\end{itemize}

\subsection*{Cofundador y CTO, Synacore SpA}
\begin{itemize}[left=0em]
    \item Diseñé y lideré el desarrollo de soluciones de software para los proyectos de la empresa, incluyendo una exitosa empresa conjunta con una empresa italiana para entrar en el mercado chileno.
\end{itemize}

\subsection*{Ingeniero de Aseguramiento de Calidad de Software, Observatorio ALMA}
\begin{itemize}[left=0em]
    \item Lideré la preparación de tres subdepartamentos para auditorías internas de calidad, asegurando el cumplimiento de las normas ISO 9001:2005.
    \item Gestioné con éxito el proceso de auditoría y lideré la resolución de todos los problemas identificados, produciendo e implementando políticas y procedimientos faltantes para cumplir con los estándares de calidad.
    \item Produje informes de análisis de fallos para hacer seguimiento de los tiempos de inactividad de las observaciones debidos a problemas de software o hardware, contribuyendo a la mejora de las operaciones del observatorio.
\end{itemize}

\section*{Educación}
\begin{itemize}[left=0em]
    \item Universidad Adolfo Ibáñez, Santiago - Magíster en Gestión Tecnológica y Emprendimiento. En Progreso.
    \item Babson College, Boston, MA - Programa Babson Build, verano de 2023.
    \item Universidad Tecnológica Metropolitana, Santiago - Licenciado en Ciencias de la Computación 2017.
    \item Universidad Tecnológica Metropolitana, Santiago - Ingeniero en Ciencias de la Computación 2009.
\end{itemize}

\section*{Habilidades Laborales}
\begin{itemize}[left=0em]
    \item Idiomas: Español (nativo), Inglés (avanzado), Alemán (principiante - A1.1).
    \item Lenguajes de Programación: C/C++, Java, PHP, C#, Python, Fortran.
    \item Bases de Datos: MySQL, PostgreSQL, SQL Server.
    \item Sistemas Operativos: Linux (Debian, RedHat, CentOS, Ubuntu, Arch, Gentoo), Mac OS X, Windows, FreeBSD.
    \item Otros: Scripting en Shell, R, minería de datos, aprendizaje automático, visión por computadora.
\end{itemize}

\section*{Otra Experiencia Relevante}
\subsection*{Asistente de Cátedra, Universidad Tecnológica Metropolitana}
\begin{itemize}[left=0em]
    \item Organicé clases auxiliares para la asignatura de Programación en la Carrera de Ciencias de la Computación y diseñé las tareas prácticas para los estudiantes.
\end{itemize}

\subsection*{Asistente de Investigación, Universidad Tecnológica Metropolitana}
\begin{itemize}[left=0em]
    \item Configuré y mantuve servidores Linux para un clúster de alto rendimiento utilizado en cómputo científico. Diseñé y ejecuté proyectos de investigación en el Laboratorio de Investigación Ambiental del Departamento de Física de la Universidad Tecnológica Metropolitana.
\end{itemize}

\section*{Otros Cursos Relevantes}
\begin{itemize}[left=0em]
    \item Lingoda - Curso de Alemán Nivel A1 - 2020.
    \item Instituto Chileno Británico - Cursos Regulares de Inglés para Adultos - 2018.
    \item Udemy - Programación en C# con Visual Studio 2017 - 2018.
    \item Udemy - Creación de Sistemas con C# y el Marco .netCore - 2017.
    \item SofO2, Santiago - Esenciales de Pruebas de Software - 2012.
    \item TÜV Rheinland, Santiago - Auditor Líder de Sistema de Gestión de Calidad ISO 9001:2005 - 2011.
    \item Universidad Austral de Chile - Inglés para Propósitos Globales - 2010.
\end{itemize}

\section*{Otros Intereses}
Soy bajista en una banda de rock local y he compuesto algunas de las canciones para el primer álbum de estudio. Además, me gusta estudiar todo lo relacionado con la música, el ritmo, la teoría, el canto y el lenguaje musical, así como la tecnología aplicada a la música.

\end{document}
